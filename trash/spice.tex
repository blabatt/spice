\section{\href{http://ngspice.sourceforge.net/docs/ngspice-manual.pdf}{SPICE}}




%%%%%%%%%%%%%%%%%%%%%%%%%%%%%%%%%%%%%%%
\subsection*{General Syntax}
\begin{lstlisting}
.TITLE PNP Power Switch
\end{lstlisting}

\mycolX{40mm}{\code{.TITLE PNP Power Switch}} \$ title \\
\mycolX{40mm}{\code{* Simulating a BJT switch}} \$ line comment \\
\mycolX{40mm}{\code{.GLOBAL gnd vcc}} \$ global elements \\
\mycolX{40mm}{\code{.INCLUDE /my/file}} \$ import file \\
\mycolX{40mm}{\code{.LIB /my/file mod1}} \$ import module \\
\mycolX{40mm}{\code{.temp 27}} \$ set temperature \\
\mycolX{40mm}{\code{.param pvcc=5}} \$ variables \\
\mycolX{40mm}{\code{.param vmax=\{1 + pvcc\}}} \$ expressions \\
\mycolX{40mm}{\code{.func foo(a,b) = {a + b}}} \$ function \\
\mycolX{40mm}{\code{RF2=1K \$ Gain set by RF2 }} \$ in-line cmnt \\
\mycolX{40mm}{\code{.END}} \$ end SPICE \\

\textit{Conditionals used for model-selection:}\\
{\scriptsize
\mycolX{60mm}{\code{.if(m0==1) .model N1 NMOS level=49 Version=3.1}} \\
\mycolX{60mm}{\code{.elseif(m1==1) .model N1 NMOS level=49 Version=3}} \\
\mycolX{60mm}{\code{.else .model N1 NMOS level=49 Version=3.3.0}} \\
\mycolX{60mm}{\code{.endif}} \\
}

\textit{Scale factors $\in$ [T|G|M|K|m|u|n|p|f] suffixed directly following number.}\\

\textit{Expressions can include many of the normal functions: trig (\texttt{sin}, \texttt{cos}, \texttt{tan}, \texttt{acos}, \texttt{asin}, \texttt{atan}), hyperbolic (\texttt{sinh}, \texttt{cosh}, \texttt{acosh}, \texttt{asinh}, \texttt{atanh}), exponential (\texttt{exp}, \texttt{ln}, \texttt{log}, \texttt{log10}), misccellaneous (\texttt{abs}, \texttt{sqrt}, \texttt{u}, \texttt{u2}, \texttt{uramp}, \texttt{floor}, \texttt{ceil}, \texttt{i}).} \\


%%%%%%%%%%%%%%%%%%%%%%%%%%%%%%%%%%%%%%%
\subsection*{Net Lists}

\textit{Each entry consists of an \say{instance} of a device element, indicating a type (first letter of \texttt{DEVICE}), nodal connections, and element-specific parameters. In general:}\\
{\scriptsize
\mycolX{40mm}{\code{DEVICE [NODE]+ VALUE [MODEL]? [PARAM]*}}} \\[2mm]


\textit{Resistor, capacitor, inductor, switches entered as follows:}\\

{\scriptsize 
\mycolX{60mm}{\textit{\code{RXXXXXXX n+ n- [resistance|r=]value [ac=val]}}}\\
\mycolX{60mm}{\textit{\code{+  [m=val] [scale=val] [temp=val] [dtemp=val] }}} \\
\mycolX{60mm}{\textit{\code{+ [tc1=val] [tc2=val] [noisy=0|1]}}} \\
\mycolX{40mm}{\code{R2 5 7 1K ac=2K}} \$ 1k resistor\\[1mm]

\mycolX{40mm}{\textit{\code{CXXXXXXX n+ n- [value] [mname] [m=val] }}} \\
\mycolX{40mm}{\textit{\code{+[scale=val] [temp=val] [dtemp=val] [tc1=val] }}} \\
\mycolX{40mm}{\textit{\code{+[tc2=val] [ic=init\_condition]}}} \\
\mycolX{40mm}{\code{COSC 17 23 10U IC=3V}} \$ 10$\mu$ capacitor \\[1mm]

\mycolX{40mm}{\textit{\code{LYYYYYYY n+ n- [value] [mname] [nt=val] }}} \\
\mycolX{40mm}{\textit{\code{+[m=val] [scale=val] [temp=val] [dtemp=val] }}} \\
\mycolX{40mm}{\textit{\code{+[tc1=val] [tc2=val] [ic=init\_condition]}}} \\
\mycolX{40mm}{\code{LSHUNT 23 51 10U IC=15.7MA}} \$ inductor \\[1mm]

\mycolX{40mm}{\textit{\code{SXXXXXXX N+ N- NC+ NC- MODEL [ON][OFF]}}} \\
\mycolX{40mm}{\textit{\code{WYYYYYYY N+ N- VNAM MODEL [ON][OFF]}}} \\
\mycolX{40mm}{\code{s1 1 2 3 4 switch1 ON}} \$ switch \\
\mycolX{40mm}{\code{wreset 5 6 vclck lossyswitch OFF}} \$ I-controlled \\[1mm]

\mycolX{40mm}{\textit{\code{[I|V]XXXXXXX N+ N- [[DC] DC/TRAN VALUE] }}} \\
\mycolX{40mm}{\textit{\code{+[AC [ACMAG [ACPHASE]]] }}} \\
\mycolX{40mm}{\textit{\code{+[DISTOF1 [F1MAG [F1PHASE]]] }}} \\
\mycolX{40mm}{\textit{\code{+[DISTOF2 [F2MAG [F2PHASE]]] }}} \\
\mycolX{45mm}{\code{VIN 13 2 0.001 AC 1 SIN(0 1 1MEG)}} \$ Indep. V-source \\
\mycolX{52mm}{\code{ISRC 23 21 AC 0.333 45.0 SFFM(0 1 10K 5 1K)}} \$ I-source \\[1mm]
\textit{Note: ind. source input can be defined by a function $\in$: \texttt{pulse}, \texttt{sin}, \texttt{exp}, \texttt{pwl}, \texttt{sffm}, \texttt{am}, \texttt{trnoise}, \texttt{trrandom}, \texttt{external}.}\\

\mycolX{40mm}{\textit{\code{[E|G]XXXXXXX N+ N- NC+ NC- VALUE [m=val]}}} \\
\mycolX{40mm}{\textit{\code{[F|H]XXXXXXX N+ N- VNAM VALUE [m=val]}}} \\
\mycolX{40mm}{\code{E1 2 3 14 1 2.0}} \$ VCVS \\
\mycolX{40mm}{\textit{\code{F1 13 5 VSENS 5 m=2}}} \$ CCCS \\[1mm]

\mycolX{40mm}{\textit{\code{DXXXXXXX n+ n- mname [area=val] [m=val] }}} \\
\mycolX{40mm}{\textit{\code{+ [pj=val] [off] [ic=vd] [temp=val] [dtemp=val]}}} \\
\mycolX{40mm}{\textit{\code{DBRIDGE 2 10 DIODE1}}} \$ Diode \\[1mm]

\mycolX{40mm}{\textit{\code{QXXXXXXX nc nb ne [ns] [tj] mname [area=val] }}} \\
\mycolX{40mm}{\textit{\code{+ [areac=val] [areab=val] [m=val] [off] }}} \\
\mycolX{40mm}{\textit{\code{+ [ic=vbe,vce] [temp=val] [dtemp=va]}}} \\
\mycolX{40mm}{\code{Q23 10 24 13 QMOD IC=0.6, 5.0}} \$ BJT \\[1mm]

\mycolX{40mm}{\textit{\code{MXXXXXXX nd ng ns nb mname [m=val] [l=val] [w=val]}}} \\
\mycolX{40mm}{\textit{\code{+ [ad=val] [as=val] [pd=val] [ps=val] [nrd=val]}}} \\
\mycolX{40mm}{\textit{\code{+ [nrs=val] [off] [ic=vds, vgs, vbs] [temp=t]}}} \\
\mycolX{40mm}{\code{M31 2 17 6 10 MOSN L=5U W=2U}} \$ MOSFET \\[1mm]

% JFET
% MESFET

}


%%%%%%%%%%%%%%%%%%%%%%%%%%%%%%%%%%%%%%%
\subsection*{Control (Batch Mode)}

\mycolX{40mm}{\code{.nodeset v(12)=4.5 v(4)=2.2}} \# iter. start guess\\
\mycolX{40mm}{\code{.ic v(11)=5 v(4)=-5 v(2)=2.2}} \# set init. cond\textquotesingle s\
\mycolX{40mm}{\code{.control}} \# enter \\
\mycolX{40mm}{\code{\dots}} \# control section \\
\mycolX{40mm}{\code{.endc}} \# leave \\

\textit{When in a control section, you can use normal programmatic control techniques like }\texttt{label},\texttt{goto},\texttt{continue},\textit{ and }\texttt{break}\textit{, in addition to loops:}\\
\mycolX{40mm}{\code{while <cond> \dots end}} \# while \\
\mycolX{40mm}{\code{repeat <num> \dots end}} \# repeat\\
\mycolX{40mm}{\code{dowhile <cond> \dots end}} \# dowhile\\
\mycolX{40mm}{\code{foreach <var> <val> \dots end}} \# foreach\\
\mycolX{40mm}{\code{if <cond> \dots else \dots end}} \# if-then\\


\mycolX{40mm}{\code{.options reltol=.005}} \# set an option $\in$\\
{\scriptsize
\begin{tabular}{l l l l}
    \underline{Gen/Elem} & \underline{DC/OP} & \underline{AC/Tran} \\[1mm]
    {[NO]}ACCT & ABSTOL & AUTOSTOP   &   \\
    NOINIT  & GMIN      & CHGTOL  &   \\
    LIST    & ITIL[1|2] & CONVSTEP  &   \\
    NOMOD   & NOOPITER  & CONVABSTEP  &   \\
    NOPAGE  & PIVREL    & INTERP  &   \\
    NODE    & PIVTOL    & ITIL[3-6]  &   \\
    OPTS    & RELTOL    & MAXEVITER  &   \\
    SEED    & RSHUNT    & MAXOPALTER  &   \\
    TEMP    & VNTOL     & MAXORD  &   \\
    TNOM    &           & METHOD  &   \\
    WARN    &           & NOOPALTER  &   \\
    BADMOS3 &           & RAMPTIME  &   \\
    DEFA[D|S] &           & SRCSTEPS  &   \\
    DEF[L|W] &           & TRTOL  &   \\
    SCALE &           & XMU  &   \\
\end{tabular}}




%%%%%%%%%%%%%%%%%%%%%%%%%%%%%%%%%%%%%%%
\subsection*{Interactive Session}

\textit{General list of commands:}\\
{\footnotesize \begin{multicols}{4}\begin{itemize}[leftmargin=0,label={}]
    \item ac
    \item alias
    \item alter
    \item altermod
    \item alt\textquotesingle rparam
    \item asciiplot
    \item aspice
    \item bug
    \item cd
    \item cdump
    \item circbyline
    \item codemodel
    \item compose
    \item dc
    \item define
    \item deftype
    \item delete
    \item destroy
    \item devhelp
    \item diff
    \item display
    \item echo
    \item edit
    \item edisplay
    \item eprint
    \item eprvcd
    \item fft
    \item fourier
    \item getcwd
    \item gnuplot
    \item hardcopy
    \item help
    \item history
    \item inventory
    \item iplot
    \item jobs
    \item let
    \item linearize
    \item listing
    \item load
    \item mc\_source
    \item meas
    \item m[r]dump
    \item noise
    \item op
    \item option
    \item plot
    \item pre\_<cmd>
    \item print
    \item psd
    \item quit
    \item rehash
    \item remcirc
    \item remzerov\textquotesingle c
    \item reset
    \item reshape
    \item resume
    \item rspice
    \item rusage
    \item save
    \item sense
    \item set
    \item setcs
    \item setcirc
    \item setplot
    \item setscale
    \item setseed
    \item settype
    \item shell
    \item shift
    \item show
    \item showmod
    \item snload
    \item snsave
    \item source
    \item spec
    \item status
    \item step
    \item stop
    \item strcmp
    \item sysinfo
    \item tf
    \item trace
    \item tran
    \item transpose
    \item unalias
    \item undefine
    \item unlet
    \item unset
    \item version
    \item where
    \item wrdata
    \item write
    \item wrs2p
\end{itemize}\end{multicols}}


%{\begin{multicols}{3}\begin{itemize}[leftmargin=0,label={}]
%    \item 
%\end{itemize}\end{multicols}}


%%%%%%%%%%%%%%%%%%%%%%%%%%%%%%%%%%%%%%%
\subsection*{Analyses}

\textbf{AC} \textit{\underline{dec}ade, \underline{oct}ave, \underline{lin}ear variation for \underline{n}umb of pts:}\\
\mycolX{45mm}{\code{\textit{.ac [dec|oct|lin] n fstart fstop}}} \# {\scriptsize ac analysis} \\
\mycolX{45mm}{\code{.ac dec 10 1K 100MEG}} \# {\scriptsize $f \in [1e3,1e8]$ }\\[1mm]

\textbf{DC} \textit{Sweep a source, resistor, or temperature:}\\
\mycolX{40mm}{\code{\textit{.dc srcnam vstart vstop vincr}}} \# {\scriptsize dc analysis}\\
\mycolX{40mm}{\code{.dc VIN 0.25 5.0 0.25}} \# {\scriptsize sweep $V_{IN} \in [.25,5]$}\\
\mycolX{40mm}{\code{.dc TEMP -15 75 5}} \# {\scriptsize temp sweep} \\

\textbf{Distortion} \textit{:}\\
\mycolX{40mm}{\code{\textit{.disto dec nd fstart fstop}}} \# {\scriptsize }\\
\mycolX{40mm}{\code{.disto dec 10 1kHz 100MEG}} \# {\scriptsize }\\

\textbf{Noise} \textit{ :}\\
\mycolX{50mm}{\code{\textit{.noise v(output) src [dec|lin|oct] pts fstart fstop}}}\\
\mycolX{55mm}{\code{.noise v(5) VIN dec 10 1kHz 100MEG}} \# {\scriptsize }\\

\textbf{Operating Point} \textit{:}\\
\mycolX{40mm}{\code{\textit{.op}}} \# {\scriptsize }\\[1mm]

\textbf{Pole-Zero} \textit{ :}\\
\mycolX{55mm}{\code{\textit{.pz nd1 nd2 nd3 nd4 [vol|cur] [pol|zer|pz]}}} \\
\mycolX{35mm}{\code{.pz 1 0 3 0 cur pol}} \# {\scriptsize pole-only for currents}\\
\mycolX{35mm}{\code{.pz 1 0 3 0 vol pz}} \# {\scriptsize poles \& zeros}\\

\textbf{Sensitivity} \textit{ :}\\
\mycolX{40mm}{\code{\textit{.sens outvar}}} \# {\scriptsize }\\
\mycolX{40mm}{\code{\textit{+ [dec|oct|lin] n fstart fstop]}}} \\
\mycolX{40mm}{\code{.sens V(1,OUT)}} \# {\scriptsize }\\
\mycolX{40mm}{\code{.sens V(OUT) ac dec 10 100 100k}} \\

\textbf{Transfer Function} \textit{ :}\\
\mycolX{30mm}{\code{\textit{.tf outvar insrc}}} \# {\scriptsize }\\
\mycolX{30mm}{\code{.tf v(5, 3) VIN}} \# {\scriptsize ratio $\frac{V_{5}}{V_{IN}}$}\\
\mycolX{30mm}{\code{.tf i(VLOAD) VIN}} \# {\scriptsize }\\

\textbf{Transient} \textit{Study temporal transient, optionally \underline{u}sing \underline{i}nitial \underline{c}onditions:}\\
\mycolX{40mm}{\code{\textit{.tran tstep tstop [tstart [tmax]] [uic]}}} \\
\mycolX{40mm}{\code{.tran 1ns 100ns}} \# {\scriptsize $100ns$ transient}\\
\mycolX{40mm}{\code{.tran 1ns 1us 500ns}} \# {\scriptsize start later}\\

\textbf{Periodic Steady State} \textit{[experimental] calculate indicated harmonics:}\\
\mycolX{60mm}{\code{\textit{.pss gfreq tstab oscnob psspoints }}} \\
\mycolX{60mm}{\code{\textit{+ harms sciter steadycoeff [uic]}}} \\
\mycolX{60mm}{\code{.pss 150 200e-3 2 1024 11 50 5e-3 uic}} \\
\mycolX{60mm}{\code{.pss 624e6 500n bout 1024 10 100 5e-3 uic}} \\




%%%%%%%%%%%%%%%%%%%%%%%%%%%%%%%%%%%%%%%
\subsection*{Output}
\textit{$\exists$ several means of getting data out: \texttt{.meas} (interactive-mode only), \texttt{.plot}, \texttt{.print}, \texttt{.four}, \texttt{.probe}, \texttt{.save}. .meas is used in interactive, .plot, .print, .four, in batch, and .probe and .save in both modes.}


\mycolX{40mm}{\code{\textit{.save [vector]+}}} \\
\mycolX{40mm}{\code{.save i(vin) node1 v(node2)}} \# {\scriptsize save 3 nodes}\\
\mycolX{40mm}{\code{.save \@m1[id] vsource\#branch}} \# {\scriptsize internal device data}\\[1mm]

\mycolX{30mm}{\code{\textit{.print prtype [ov]+}}} \# {\scriptsize prt \underline{o}utput \underline{v}ariable \say{vectors}}\\
\mycolX{40mm}{\code{.print tran v(4) i(vin)}} \\
\mycolX{40mm}{\code{.print dc v(2) i(vsrc) v(2, 7)}} \# {\scriptsize volt-$\Delta$, $v(2)-v(7)$}\\
\mycolX{40mm}{\code{.print ac vm(4, 2)}} \# {\scriptsize volt-$\Delta$ \underline{m}agnitude}\\[1mm]

\mycolX{45mm}{\code{\textit{.plot pltype [ov1 [(plo1, phi1)]?]+}}} \# {\scriptsize }\\
\mycolX{40mm}{\code{.plot dc v(4) v(5) v(1)}} \# {\scriptsize no \say{plo}, \say{phi}}\\
\mycolX{40mm}{\code{.plot dc emit base out}} \# {\scriptsize using node names}\\
\mycolX{40mm}{\code{.plot db(coll) ph(coll)}} \# {\scriptsize Bode-style plot}\\
\mycolX{40mm}{\code{.plot tran v(17, 5) (2, 5)}}  {\scriptsize }\\[1mm]

\mycolX{30mm}{\code{\textit{.four freq [ov]+}}} \# {\scriptsize fourier analysis}\\
\mycolX{30mm}{\code{.four 100K v(5)}} \# {\scriptsize  fundamental \texttt{freq} $100k$}\\[1mm]

\mycolX{40mm}{\code{\textit{.probe [vector]+}}} \# {\scriptsize save to rawfile}\\
\mycolX{40mm}{\code{.probe i(vin) input output}} \\[1mm]

\texttt{.meas} \textit{ defaults to extracting time, use \texttt{FIND} for another variable. Store measurement in \texttt{result} parameter, and transform it with specified \underline{FUNC}tion $\in$ {\scriptsize [AVG|MAX[\_AT]|MIN[\_AT]|PP|RMS|DERIV|INTEG]}. Snapshot data at specific time with \texttt{WHEN}, or bookend a range of time (generally, the abscissa) with triggers using \texttt{\underline{TRIG}}ger .. \texttt{\underline{TARG}}et, or at specified times using \texttt{FROM} .. \texttt{TO}.}\\
{\scriptsize \mycolX{40mm}{\code{\textit{.MEAS [DC|AC|TRAN|SP] result [FUNC]?}}}\\
\mycolX{40mm}{\code{\textit{+ [[TRIG|TARG] trig\_var VAL=val [TD=td]? }}} \\
\mycolX{40mm}{\code{\textit{+ [CROSS=\# | CROSS=LAST]? [RISE=\# | RISE=LAST]?}}} \\
\mycolX{40mm}{\code{\textit{+ [FALL=\# | FALL=LAST]? [AT=time]?}}}} \\
{\scriptsize 
\mycolX{40mm}{\code{.measure tran tdiff TRIG v(1) VAL=0.5 RISE=1 }}\\
\mycolX{40mm}{\code{+ TARG v(1) VAL=0.5 RISE=2}}\\
\mycolX{40mm}{\code{.measure tran teval WHEN v(2)=0.7 CROSS=LAST}} \\
\mycolX{40mm}{\code{.measure tran teval WHEN v(2)=v(1) RISE=LAST}} \\
\mycolX{40mm}{\code{.measure tran yeval FIND v(2) WHEN v(1)=-0.4 FALL=LAST}} \\
\mycolX{40mm}{\code{.meas tran out\_slew trig v(out) val=\textasciigrave 0.2*vp\textasciigrave rise=2 }} \\
\mycolX{40mm}{\code{+ targ v(out) val=`0.8*vp` rise=2}} \\
}




%%%%%%%%%%%%%%%%%%%%%%%%%%%%%%%%%%%%%%%
\subsection*{Subcircuits}

\textit{Afford hierarchically nestable modeling (though flattened upon compilation). Can expose an arbitrary \# of nodes. For example, a voltage divider:}\\
\mycolX{35mm}{\code{.subckt vdivide 1 2 3}} \$ 3 ports \\
\mycolX{35mm}{\code{r1 1 2 10K}} \$ 10K resistor \\
\mycolX{35mm}{\code{r2 2 3 5K}} \$ 5K resistor \\
\mycolX{35mm}{\code{.ends vdivide}} \$ exclude to end all \\[2mm]
\textit{Can parameterize subckts, here with rval, cval:}\\
\mycolX{40mm}{\code{.subckt myfilter in out rval=100k  cval=100nF}} \\


%%%%%%%%%%%%%%%%%%%%%%%%%%%%%%%%%%%%%%%
\subsection*{Models}

\textit{General format and example:}\\
\mycolX{60mm}{\code{.model [NAME] [TYPE] ([PARAM]* )}}  \\
\mycolX{60mm}{\code{.model MOD1 npn (bf=50 is=1e-13 vbf=50)}}  \\

\textit{Model Types $\in$: }\\
\begin{tabular}{l l}
Code & Type \\
R & Resistor\\
C & Capacitor \\
L & Inductor \\
SW & V-controlled Switch\\
CSW & I-controlled Switch \\
URC & Uniform RC Model \\
LTRA & Lossy Trans. Line\\
D & Diode \\
NPN & NPN BJT \\
PNP & PNP BJT \\
NJF & N-channel JFET\\
PJF & P-channel JFET \\
NMOS & N-channel MOSFET\\
PMOS & P-channel MOSFET\\
NMF & N-channel MESFET\\
PMF & P-channel MESFET\\
VDMOS & Power MOS\\
\end{tabular}



%%%%%%%%%%%%%%%%%%%%%%%%%%%%%%%%%%%%%%%
\subsection*{API}
\textit{Include \texttt{sharedspice.h} in client code. Compile ngspice with -{}-with-ngshared to yield ngshared.lib, which you link to from client code.}\\

\textit{Callbacks $\in$:}
\begin{multicols}{2}
{\footnotesize 
\begin{itemize}[label={}]
    \item SendChar 
    \item SendStat
    \item ControlledExit
    \item SendData
    \item SendInitData
    \item BgThreadRunning
    \item SendEvtData
    \item SendInitEvtData
\end{itemize}
}
\end{multicols}

\textit{API commands prefaced with } \texttt{ngSpice\_} \textit{, suffixed with:}
\begin{multicols}{2}
{\footnotesize 
\begin{itemize}[label={}]
    \item Init
    \item Init\_Sync
    \item running
    \item Command
    \item Get\_Vec\_Info
    \item CurPlot
    \item AllPlots
    \item AllVecs
    \item SetBkpt
    \item Init\_Evt
    \item Get\_Evt\_NodeInfo
    \item All\_Evt\_Nodes
\end{itemize}
}
\end{multicols}

\textit{Data structs to retrieve simulation data $\in$} \texttt{vecinfo}, \texttt{vector\_info}, \texttt{vecinfoall}. \\





